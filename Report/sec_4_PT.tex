As we have discussed previously, the problem of learning the model for an RBM can be boiled down to effectively we can approximate $\E_P\left(\frac{\partial E}{\partial \theta}\right)$. In order to do this, we had to resort to Gibbs Sampling. 
Gibbs sampling performs local updates to generate a Markov Chain with stationary distribution $P(\textbf{v},\textbf{h})$ which is complex and multi-modal. Despite the theoretical guaranteed convergence of Gibb's sampling, the local nature of the updates means that chain can have a very difficult time traversing the modes of $P$ and  has a tendency to get trapped exploring the local areas of our sample space of high probability. This means that certain modes are over-represented, while other are under-represented and leading to a poor approximation of our expectation.

Parallel tempering (PT) is a class of Monte Carlo methods first introduced by Geyer in \cite{geyer1991markov} that improve the miximng our our MCMC chain, by simultaneuously running $N$ MCMC chains in parallel but at different temperatures. The higher temperature chains focus on exploration of the state space, while the room temperature chain focus on the accuracy of samples. In order for there to be communication between the hot and room temperature chains, we periodically propose swaps between the chains in such a what that does not propose 


\subsection{Parallel Tempering}\label{sec_PT}
Let $P$ be a Gibbs distribution over some sample space $\Omega$ of the form,
\[P(x)=\frac{1}{Z} \exp(-E(x)).\]
Note that $E$ can be that of an RBM but does not have to be. We define our inverse-temperature space as $\mathcal{B}=[\beta_{\min},1]$, for $0\leq \beta_{\min}<1$. Given $\beta\in\mathcal{B}$, let $P^{(\beta)}$ the probability distribution identified with the un-normalized density $P(x)^\beta$, which we can write
\[P^{(\beta)}(x)= \frac{1}{Z(\beta)} \exp(-\beta E(x)),\]
where $Z(\beta)$ is the partition function. 

Given a sequence of inverse temperatures $0\leq\beta_{\min}= \beta_N<\cdots<\beta_{1}<\beta_0=1$, we define the measure $\tilde{P}$ on $\Omega^{N+1}$ by,
\[\tilde{P}=P^{(\beta_0)}\times\cdots\times P^{(\beta_N)}.\]
In parallel tempering we construct a Markov chain $\tilde{X}_n=(X_n^{0},\cdots, X^{N}_n)$ on $\Omega^N$ with stationary distribution $\tilde{\pi}$. We update $X_n$ by alternate between two different type of dynamics: exploration and communication. 

During the exploration phase, we allow for each $X_n^i$ to independently explore $\Omega$ according to $P^{(\beta_i)}$ via our favourite MCMC algorithm such as Gibbs Sampling. Since MCMC updates of each component $i$ leaves $P^{(\beta_i)}$ invariant, we have the exploration phase leaves $\tilde{P}$ invariant.

During the communication phase, we propose a swap between the $\beta_i$ and $\beta_j$ components of $\tilde{X}_n$, in other words we swap $X^i_n$ and $X^j_n$ in $\tilde{X}_n$. To simplify notation, given $x=(x^0,\cdots,x^i,\cdots,x^j,\cdots, x^N)$, we will define $x_{(i,j)}=(x^0,\cdots,x^j,\cdots,x^i,\cdots,x^N)$. So the proposed state during the communication phase is $(\tilde{X}_n)_{(i,j)}$. In order to keep $\tilde{X}_n$ stationary with respect to $\tilde{P}$, we can make this update reversible by accepting this proposed swap according to the Metropolis acceptance,
\begin{align*}
\alpha
&=1\wedge\frac{\tilde{P}((\tilde{X}_n)_{(i,j)})}{\tilde{P}(\tilde{X}_n)}\\
&=1\wedge\frac{P^{(\beta_i)}(X_n^j)P^{(\beta_j)}(X_n^i)}{P^{(\beta_i)}(X_n^i)P^{(\beta_j)}(X_n^j)}\\
&=1\wedge\exp\left[(\beta_j-\beta_i)(E(X_n^j)-E(X_n^i))\right]\numberthis\label{accept_ratio}
\end{align*}
Since the likelihood of acceptance is proportional to the change in temperature, we usually restrict to our swaps to nearest neighbours during the communication phase.

We refer to the sequence $(X_n^{i})_{n\geq0}$ as the $i$-th chain, the sequence of states at a particular temperature $\beta_i$. During the communication phase, we can view a proposed state as a swap between two states at while fixing the temperature of the chain. Alternatively we can view the communication phase as increase or decreasing the temperature of particular state. We will call this sequence of states with varying temperatures as a replica. In particular, the $i$-th replica is the sequence $(X_n^{\eta^i_n})$, where $\eta^i_0=i$ and $\eta^i_n$ represents the temperature of state $X_n$ at time $n$. The $i$-th replica follows the trajectory of the $X_0^{i}$ state through the state space as it's temperature changes. A change in $\eta^i_n$ indicates a successful swap in the communication phase at time $n$.

\subsubsection{Temperature spacing}
The mixing of our MCMC algorithm during the exploration phase improves as $\beta$ decreases as $P^{(\beta)}(x)$ ``flattens'' our and makes it modes less pronounces. The temperature spacing is important as the acceptance probability of a proposed swap depends on the change in inverse-temperature as seen in \eqref{accept_ratio}. So if the change in inverse-temperature during a swap is small, we are far more likely to accept, but we will not be exploring a significantly better mixing chain. Alternatively, if the change in inverse-temperature is large, we will potentially swap with a better mixing chain, but the acceptance ratio in \eqref{accept_ratio} will be potentially very small. We want to chose the temperature spacing in such a way that maximizes the speed we can communicate information from the high temperature chains to the room temperature chains.

In the high-dimensional setting, \cite{atchade_towards_2011} seem suggest that we pick our temperature spacing in such a way that the acceptance probability of each swap is approximately 23.4\%. However this optimal temperature spacing is difficult to attain and is seen as more of a theoretical guideline. In practice, a geometric spacing can be a good temperature spacing for many applications of interest \cite{atchade_towards_2011},\cite{kofke2002acceptance}. For the purpose of this project we will assume $\beta_n=\gamma^n$ for some $\gamma\in (0,1)$.
\subsubsection{Problems of Reversibility}
In the classical parallel tempering algorithm, because of the detailed balance condition for the swaps, the replicas must be proposed a swap that can potentially raise or lower the temperature. This forces the temperature of the replica to traverse the inverse-temperature space $\mathcal{B}$ in a random walk fashion and leads to diffusive behaviour. It takes roughly $O(N^2)$ number of swap attempts  to communicate information between $N$ chains \cite{diaconis2000analysis}. This is not favourable as the goal is to have the information from the high temperature states reach room temperature as quickly as possible, without getting distracted along the way. 

In order to eliminate this diffusive behaviour and speed up the rate of communication between states, we need to construct swap proposals that are non-reversible. We will discuss two such methods called lifted parallel tempering (LPT) outlined in \cite{wu_irreversible_2017} and the closely related deterministic even/odd algorithm which we will refer to as LTPD as discussed in \cite{lingenheil2009efficiency}. 

\subsection{Lifted Parallel Tempering}
Before we discuss LPT, we will need to introduce the Generalized Metropolis Hastings framework as outlined in \cite{stoltz2010free}.
\subsubsection{Generalized Metropolis-Hastings}\label{sec_GMH}
Suppose we have a probability measure $P$ over measurable space $(\Omega,\mathcal{F})$ and a function $S:(\Omega,\mathcal{F})\to (\Omega,\mathcal{F})$ is a measure preserving involution for $P$. If $Q(y|x)$ is a Metropolis-proposal kernel, then we will construct a Markov transition kernel as follows. Suppose we are at state $x$, we propose a state $y$ via the kernel $\tilde{Q}$ given by $\tilde{Q}(y|x)=Q(S(dy)|x)$ and accept $y$ as a new state with probability 
\[\alpha(x,y)=\frac{\pi(y)\tilde{Q}(y|x)}{\pi(x)\tilde{Q}(x|y)}.\]
This induces the reversible Metropolis transition kernel, 
\[\tilde{K}(y|x)=\alpha(x,y)\tilde{Q}(y|x)+\left(1-\int\alpha(x,y)\tilde{Q}(y|x)dy\right)\delta_x(y).\]
which has stationary distribution $P$. Since $\tilde{K}$ and $S$ preserve $P$, then so does $K\equiv S\circ \tilde{K}$, which is given by,
\begin{align*}
K(y|x)=\alpha(x,y)Q(y|x)+\left(1-\int\alpha(x,y)Q(y|x)dy\right)\delta_{S(x)}(y).
\end{align*}
Effectively, we make a proposal according to $Q$, which is accepted with probability $\alpha$ and apply $S$ if the proposal is rejected. This in general produces an non-reversible scheme as the composition of two reversible kernels is not necessarily reversible.
\subsubsection{Lifted Parallel Tempering Algorithm}
Suppose we have the same set-up as PT outlined in \S\ref{sec_PT}. The only modification we will make for LPT will be for the communication phase.

We have $N$ additional chains running simultaneously with joint distribution $\tilde{P}=\pi^{(\beta_0)}\times\dots\times\pi^{(\beta_N)}$. We denote the state at time $n$ by $X_n=(X^0_n,\dots, X^N_n)$, where $X^i$ is at temperature $\beta_i$. We will choose to make one of the replicas ``lifted'', say replica $\eta_0$.  In addition we introduce the lifted parameters $(\eta_n,\varepsilon_n)\in \{0,1,\dots,N\}\times\{\pm 1\}$. Here $\eta_n$ represents temperature index of the $\eta_0$-th replica (which we will call the ``lifted'' replica) and $\varepsilon_n$ represents the direction the replica will attempt to swap, i.e, the next proposed swap will be between $X^{\eta_n}_n$ and $X^{\eta_n+\eps_n}_n$. 

To deal with the boundary cases, note that the replica is at temperature $\beta_0$ or $\beta_N$ then it must be proposed a swap with the state at temperature $\beta_1$ and $\beta_{N-1}$ respectively. So we have
\[(\eta_n,\varepsilon_n)\in \{0,1,\dots,N\}\times\{\pm 1\}\backslash\{(0,-1),(N,1)\}\equiv \mathcal{X}. \]

Let $Z_n=(X_n,\eta_n,\epsilon_n)$ be a Markov chain in enlarged space will be $\Omega^{N+1}\times\mathcal{X}$. Suppose we are working with the deterministic proposal
\[Q(y|z)=\delta_{\phi(z)}(y)\]
given by, 
\begin{align*}
\phi(X,\eta,\eps)=
\begin{cases}
(X_{(\eta,\eta+\eps)},\eta+\eps,\eps) &   0<\eta+\eps<N,\\
(X_{(\eta,\eta+\eps)},\eta+\eps,-\eps) &  \mathrm{Otherwise}.\\
\end{cases}
\end{align*}
We also define the involution $S:\Omega^{N+1}\times\mathcal{X}$ given by,
\begin{align*}
S(X,\eta,\eps)=
\begin{cases}
(X,\eta,-\eps) &   0<\eta+\eps<N,\\
(X,\eta,\eps) &  \mathrm{Otherwise},
\end{cases}
\end{align*}
which preserves the measure $\tilde{P}\times\mathrm{Unif}(\mathcal{X})$. Together $Q$ and $S$ give us the recipe for the lifted parallel tempering algorithm via the generalized Metropolis-Hastings framework from \S\ref{sec_GMH}.

If a swap is accepted and increases (decreases) the temperature of the replica, then the next swap will also be propose an increase (decrease) in temperature. If the swap is rejected, then the next swap will propose a decrease (increase) in temperature. If the replica achieves a temperature of either reaches $\beta_0$ or $\beta_N$, then we reverse the direction of the swaps. To summarize in LPT, the lifted replica is proposes swaps with a ``momentum'' (i.e. the communication resists change of direction). It was shown in \cite{diaconis2000analysis} that this persisted swapping required $O(N)$ number of swap attempts on average to swap information between $N$ chains compared to the $O(N^2)$ number of swaps required for PT.

\subsection{Deterministic Even/Odd Algorithm}
Finally we will introduce one more variant of PT that has become quite popular in practice recently call the deterministic even/odd algorithm or deterministic lifted parallel tempering (LPTD) as outlined in \cite{lingenheil2009efficiency}. The only modification we are making from PT is changing the swap proposals during the communication phase. 

Let $\mathcal{N}=\{(i,i+1):0\leq i<N\}$, be the set of adjacent pairs. We partition $\mathcal{N}=\mathcal{E}\cup\mathcal{O}$, where
\begin{align*}
\mathcal{E}=\{(i,i+1)\in \mathcal{N}: i \text{ is even}\},\\
\mathcal{O}=\{(i,i+1)\in \mathcal{N}: i \text{ is odd}\}.
\end{align*}
In the first communication phase, we propose swaps with chains $i$ and $i+1$ for all $(i,i+1)\in\mathcal{E}$ and accept/reject them with the metropolis acceptance ratio \eqref{accept_ratio}. Let us call this reversible metropolis-kernel $K_E$. Similarly for the next communication phase we propose swaps with chains $i$ and $i+1$ for all $(i,i+1)\in\mathcal{O}$ and accept/reject them with the metropolis acceptance ratio \eqref{accept_ratio}. Let us call this reversible metropolis-kernel $K_O$. We then repeat this process by continuing to alternate between even and odd proposal kernels. 

This is a valid MCMC algorithm, since each kernel leaves $\tilde{P}$ invariant. But similar to LPT, the composition of two reversible kernels $K_O\circ K_E$ need not be reversible. In particular, the connection with LPTD is that each replica follows the trajectory of a lifted replica from LPT. In particular for the $i$-th replica, $X_n^{\eta^i_n}$, if $\eta_n^i$ is even, and a swap is successful then $\eta^i_{n+1}=\eta_n^i+1$ will be odd and will be proposed an increase in temperature. If the swap was rejected then $\eta_{n+1}^i=\eta_n^i$ will be even, and will be proposed a swap to decrease the temperature. So $\eta_n^i$ will continue to increase until a rejection occurs, in which case the direction will be reversed. An identical argument shows the same holds when $\eta_n^i$ is odd. This mean $i$-th replica for LPTD is followd the exact same trajectory in the inverse-temperature space as the ``lifted'' replica in LPT.

This shows that the same advantages we expect out of LPT should hold true for LPTD. The advantage is that LPTD has multiple replicas carrying information as opposed to LPT, where there is one. It was shown in \cite{lingenheil2009efficiency} that LPTD empirically out-performs a variety of other state-of-the-art swapping schemes. This connection between LPT, and LPTD is the theoretical reason why it outperforms the competing swapping schemes.
